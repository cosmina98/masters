A review of the algorithms and datasets in the literature of the Unbounded Knapsack Problem (UKP) is presented in this master's thesis.
The algorithms and datasets used are briefly described in this work to provide the reader with basis for understanding the discussions.
Some well-known UKP-specific properties, such as dominance and periodicity, are described.
The UKP is also superficially studied in the context of pricing problems generated by the column generation approach applied to the continuous relaxation of the Bin Packing Problem (BPP) and Cutting Stock Problem (CSP).
Multiple computational experiments and comparisons are performed.
For the most recent artificial datasets in the literature, a simple dynamic programming algorithm, and its variant, seems to outperform the remaining algorithms, including the previous state-of-the-art algorithm.
The way dominance is applied by these dynamic programming algorithms has some implications for the dominance relations previously studied in the literature.
In this master's thesis we defend that choosing sets of artificial instances has defined what was considered the best algorithm in previous works.
We made available all codes and datasets referenced in this master's thesis.
