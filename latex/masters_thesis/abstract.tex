A review of the algorithms and datasets in the literature of the Unbounded Knapsack Problem (UKP) is presented.
The algorithms and datasets used are briefly described in this work, to give the reader a base for understanding its discussions.
Some well-known UKP-specific properties, as dominance and periodicity, are fully described.
The UKP is also superficially studied in the context of pricing problems generated by the column generation approach applied to the continuous relaxation of the Bin Packing Problem (BPP) and Cutting Stock Problem (CSP).
Multiple computational experiments and comparisons are performed.
For the most recent artificial datasets in the literature, a simple dynamic programming algorithm, and its variant, seems to outperform the remaining algorithms, including the previously state-of-the-art algorithm.
The way dominance is applied by these dynamic programming algorithms has some implications for the dominance relations previously studied in the literature.
The author defends the thesis that the choice of artificial instances defined which was considered the best algorithm in the prior work.
All codes and datasets used were made available by the author.
