\documentclass[12pt]{article}

\usepackage{sbc-template}
\usepackage{hyperref} % for theorem reference
\usepackage{cleveref} % for theorem reference
\usepackage{amsthm} % for proof
\usepackage{amssymb}
\usepackage{mathtools}
\usepackage{fixltx2e}
\usepackage{enumerate}
\usepackage[normalem]{ulem} % for sout command (striken text)

\usepackage[utf8]{inputenc}  
 
\usepackage{algorithm}
\usepackage{algpseudocode}

\newtheorem{definition}{Definition}
\newtheorem{theorem}{Theorem}
\newtheorem{lemma}{lemma}
\newtheorem{corollary}{Corollary}
\newcommand{\corollaryautorefname}{Corollary}
\newtheorem{proposition}{Proposition}
\newtheorem{claim}{Claim}
\newtheorem{axiom}{Axiom}

\let\subsectionautorefname\sectionautorefname
\let\subsubsectionautorefname\sectionautorefname

\usepackage{tikz}
\newcommand{\drawvvector}[5]{
	\edef \origin {#1}
	\edef \ymax {#2}
	\edef \rs {(#3,#4)}
	\edef \scale {#5}
	\foreach \y in {0,...,\ymax}{
		\draw [shift={\origin}](0,\y) rectangle +\rs;
		\node [scale=\scale, shift={\origin}, font=\LARGE] at (0.5, \y + 0.5) {\(\y\)};
	}
}
\newcommand{\drawhvector}[6]{
	\edef \origin {#1}
	\edef \xmax {#2}
	\edef \rs {(#3,#4)}
	\edef \scale {#5}
	\foreach \x in {0,...,\xmax}{
		\draw [shift={\origin}] (\x,0) rectangle +\rs;
		\node [scale=\scale, shift={\origin}, font=\LARGE] at (\x + 0.5, 0.5) {\(\x\)};
	}
}


\sloppy

\title{The UKP5 Method}

\author{Henrique Becker\inst{1}}

%\address{Instituto de Ciências Exatas e Geociências -- Universidade Passo Fundo (UPF)\\
%  Caixa Postal 611 -- CEP 99001-970 -- Passo Fundo -- RS -- Brasil
\address{Instituto de Informática -- Universidade Federal do Rio Grande do Sul (UFRGS) \\
  Caixa Postal 15064 -- CEP 91501-970 -- Porto Alegre -- RS -- Brasil
  \email{henriquebecker91@gmail.com}
}

\begin{document} 

\maketitle

\section{Introduction}

The objective of this work is to present an algorithm that solves the Unbounded Knapsack Problem (UKP), and to prove its correctness. We name this algorithm UKP5.

\subsection{Definition of the Problem and Chosen Notation}

In this subsection we use the chosen notation to define the UKP. We will use this notation for the rest of the article.

The UKP can be defined as follows: we are given a capacity \(c\), and a list of \(n\) items. Each item \(i \in \{1, \ldots, n\}\) has a weight value, denoted by \(w_i\), and a profit value, denoted by \(p_i\); we want to know how many of each item \(i\) we should select to get the maximal items profit sum without the items weight sum becoming greater than the capacity. The variable \(x_i\) indicates how many of item \(i\) we selected. The problem can be defined using the linear programming notation as follows:

\begin{align}
  maximize: &\sum_{i=1}^n p_i x_i\label{eq:objfun}\\
subject~to: &\sum_{i=1}^n w_i x_i \leq c\label{eq:capcons}\\
            &x_i \in \mathbb{N}_0\label{eq:x_integer}
\end{align}

The capacity \(c\), the quantity \(n\) and the weights of the items \(w_i~(\forall i.~1 \geq i \geq n)\) are all positive integers. The quantities \(x_i~(\forall i.~1 \geq i \geq n)\) are restricted to the non-negative integers, as shown in \eqref{eq:x_integer}. The profit of the items \(p_i~(\forall i.~1 \geq i \geq n)\) are all positive real numbers.

%The efficiency of an item \(i\) is the value of \(\frac{p_i}{w_i}\). We will refer to the efficiency of an item as \(e_i\). The UKP5 works faster when the items are ordered by non-decreasing efficiency (i.e. \(i < j\) iff \(\frac{p_i}{w_i} < \frac{p_j}{w_j}\), or simply \(i < j\) iff \(e_i < e_j\) (if two or more items have the same efficiency, we order them by non-decreasing weight). If the items aren't ordered the UKP5 order them first.

We use \(w_{min}\) and \(w_{max}\) to denote the smallest item weight and the biggest item weight, respectively.

\subsection{Some extra definitions}

Except when the contrary is stated, our conclusions are valid for any arbitrary UKP instance. Things that are specific from an instance will always be from the same arbitrary instance on the same context. We never compare items, solutions or capacities from two different UKP instances.

A solution is a multiset\footnote{A multiset is a set that can contain more than one of element of each type.} of the items defined by that instance. The weight of a solution is the solution items weight sum. The profit of a solution is analogue.  We will use the same notation used by items to denote the wieight and profit of solutions (i.e. the weight of a solution \(s\) is \(w_s\), and its profit is \(p_s\)).

A valid solution is a solution with weight less than or equal to the capacity established by the instance (\(w_s \leq c\)). The empty multiset is a trivial valid solution for any UKP instance. We will refer to the set of the valid solutions for an arbitrary UKP instance as \(V\), and the set of the items provided by the instance as \(I\). As we never mix items, solutions or capacities of two different instances this sintax isn't ambiguous.

An optimal solution \(s^*\) is a valid solution with the greatest profit value between all valid solutions of that instance (\(\forall s \in V.~\exists s^* \in V.~p_{s^*} \geq p_s\)). The optimal solution value is the optimal solution profit (\(opt = p_{s^*}\)). We say that one solution~\(s\) dominates another solution~\(t\) iff both are valid and \((w_s <  w_t \land p_s \geq p_t) \lor (w_s \leq w_t \land p_s > p_t)\). In the case that \(w_s = w_t \land p_s = p_t\) we say that both solutions are equivalent. All the four dominance relations presented in (CITE PYASUKP) are special cases of one solution dominating, or being dominated by, other solution.

\section{First Theorem: Dominance and optimal solutions}

If one solution is dominated by other, both can be subsets of optimal solutions\footnote{For example, see the following UKP instance: \(c = 10,~w_1 = 9,~w_2 = 10,~p_1 = p_2 = 10\); the solutions made from one item \(1\) or one item \(2\) are both optimal solutions. Other example: \(c = 12,~w_1 = 9,~w_2 = 10,~w_3 = 2,~p_1 = p_2 = 10,~p_3 = 1\).}. However, we can yet assert the following theorem.
\vspace{0.3cm}
\begin{theorem}\label{theo:dominance} 
If a solution~\(s\) is dominated by a solution \(t\), and \(s\) is a subset of an optimal solution, then \(t\) is also a subset of an optimal solution.
\end{theorem}
\begin{proof}
Lets suppose \(opt\) is an optimal solution, \(s \cup s' = opt\) and \(s\) is dominated by \(t\); lets define \(opt' = t \cup s'\); \(opt'\) will have a no lower profit and a no greater weight than \(opt\); \(opt'\) is a valid solution because its weight is less than or equal to another valid solution (\(opt\)); we have the guarantee that \(p_{opt'} \geq p_{opt}\) but also, as \(opt\) is an optimal solution, we have the guarantee that no valid solution has a greater profit than it, therefore \(p_{opt'} = p_{opt}\); by definition \(opt'\) is an optimal solution (there's no valid solution with greater profit); finally \(t \cup s' = opt' \iff t \subset opt'\), this way we prove that in this conditions \(t\) must be a subset of an optimal solution. If \(s' = \varnothing\) then both \(s\) and \(t\) are optimal solutions.
\end{proof}

\section{The proof of correctness and explanation of the algorithm}

An UKP5 algorithm correctness proof constructed directly over the algorithm optimized final code would be, in our opinion, hard to understand. We chose to follow a different path, and we will instead show a correctness proof for an initial version of the algorithm, and for each of the four subsequent algorithm optimizations we will show proof that despite of the change the code stands correct. We think that is more important that a proof is clear than that it's concise. This approach also has the beneficial side-effect of providing insight over the big performance difference that one or two lines of the algorithm code account for.

%\section{UKP0 -- The classic dynamic programming approach}

\section{UKP1 -- A quasi-exhaustive approach}

\begin{algorithm}
\caption{UKP One}\label{alg:ukp1}
\begin{algorithmic}[1]
\Procedure{UKP1}{$n, c, w, p, w_{min}, w_{max}$}
  \State \(g \gets\) array of \(c + 1 + (w_{max} - w_{min})\) positions each one initialized with \(0\)\label{ukp1:create_g}
  \State % Blank line
  \For{\(i \gets 1,~n\)}\label{begin_trivial_bounds}
    \If{\(g[w_i] < p_i\)}
      \State \(g[w_i] \gets p_i\)
    \EndIf
  \EndFor\label{end_trivial_bounds}
  \State % Blank line
  \For{\(y \gets w_{min},~c-w_{min}\)}\label{ukp1:main_ext_loop_begin}
    \If{\(g[y] == 0\)}\label{ukp1:if_equal_to_zero}
    	\State \textbf{continue}
    \EndIf
    \State % Blank line
    \For{\(i \gets 1,~n\)}\label{ukp1:main_inner_loop_begin}
      \If{\(g[y + w_i] < g[y] + p_i\)}\label{ukp1:if_better_solution_begin}
        \State \(g[y + w_i] \gets g[y] + p_i\)
      \EndIf\label{ukp1:if_better_solution_end}
    \EndFor\label{ukp1:main_inner_loop_end}
  \EndFor\label{ukp1:main_ext_loop_end}
  \State % Blank line
  \State \(opt \gets 0\)
  \For{\(y \gets c-w_{min}+1,~c\)}\label{ukp1:get_opt_loop_begin}
    \If{\(g[y] > opt\)}\label{ukp1:opt_loop_if}
      \State \(opt \gets g[y]\)
    \EndIf
  \EndFor\label{ukp1:get_opt_loop_end}
\EndProcedure
\end{algorithmic}
\end{algorithm}

The UKP1 is an algorithm for computing the UKP optimal solution value. By that, we mean that the \(opt\) variable will have the optimal solution value by the end of the algorithm execution. The proof of its correctness is broken in the three theorems bellow.

We define \textit{profit-dominance} as the dominance special case where the weight of two solutions is equal, but the profit can be distinct (i.e. \(s\) is profit-dominated by \(t\) iff \(w_s = w_t \land p_s \leq p_t\)). For an UKP instance of capacity \(c\) there's a maximum of \(c+1\) non profit-dominated solutions\footnote{For an example, for any capacity \(c\), an instance that have an element with weight \(1\) will have \(c+1\) non profit-dominated solutions (the empty solution and the solutions with \(k\)-times that one element, \(1 \leq k \leq c\)).}). These definitions will help us in the proof bellow.

The main point of the proof is that UKP1 stores in \(g\) all valid solutions, except by the ones that are profit-dominated by others. As we already saw, in \autoref{theo:dominance}, if a dominated solution would be part of an optimal solution, the dominator solution would too be part of an optimal solution. This way, we have no reason to store both dominated and dominator, and we can only store dominator solutions. As this excludes only dominated solutions, this cannot exclude all optimal solutions. Therefore the greatest profit value stored in \(g\) is by definition the optimal solution value.

\subsection{UKP1 First Theorem: All relevant solutions are generated}

%We define this theorem more accurately as: after the line~\ref{ukp1:main_ext_loop_end} of UKP1, each position \(y\) of the array \(g\) between \(w_min\) and \(c\) has biggest profit of a solution with weight exactly \(y\), or the value zero, if there's no solution that weights exactly \(y\). In mathematical notation:
%\begin{flalign}
%\forall (y \in [w_{min},c]).~\forall (s \in V ).~\exists (t \in V).&&\nonumber\\
%(g[y] > 0 \land w_s = y)& \iff ((w_t = y) \land (g[y] = p_t) \land (p_t \geq p_s))
%\end{flalign}
%\begin{flalign}
%\forall (y \in [w_{min},c]).~(g[y] = 0) \iff (\nexists (s \in V). (w_s = y))&&
%\end{flalign}

\begin{theorem}\label{theo:ukp1:all_relevant_solutions_present}
After the execution of UKP1 line~\ref{ukp1:main_ext_loop_end}, for every valid non profit-dominated solution~\(s^*\), the profit value of \(s^*\) is stored in \(g[w_{s^*}]\). In mathematical notation: \(\forall s \in V.~\exists s^* \in V.~(w_s = w_{s^*}) \land (p_s \leq p_{s^*}) \land (g[w_{s^*}] = p_{s^*})\).
\end{theorem}

\vspace{0.3cm}
\noindent
We prove this theorem by structural induction.

\subsubsection{Base case: Empty solution}
\label{ukp1:empty_solution_proof}

The empty solution is present, as \(g[0] = 0\) since line \ref{ukp1:create_g}, and is not modified after.

\subsubsection{Base case: One single item solutions}
\label{ukp1:single_item_solution_proof}

The UKP1 first loop (lines~\ref{begin_trivial_bounds}~to~\ref{end_trivial_bounds}) iterates all items and store its profits at \(g[w_i]\). If two or more items have the same weight only the greatest profit is stored. Every non profit-dominated solution made up of only one item is therefore present. The second loop (lines~\ref{ukp1:main_ext_loop_begin}~to~\ref{ukp1:main_ext_loop_end}) can change that, but only by replacing a one item solution by a multiple item solution that profit-dominates it. In this case, the one item solution was profit-dominated, so our theorem stands correct.

\subsubsection{Induction Hypothesis: \(s\) solutions}

Assume that for some valid nonempty non profit-dominated solution~\(s\), the profit value of \(s\) is stored in \(g[w_s]\). This statement is true for our base cases.

\subsubsection{Step case: \(s \cup \{i\}\) solutions}
\label{ukp1:step_case_proof}

Lets define the \(s'\) solution as \(\forall s \in V.~\forall i \in I.~(|s| > 0) \implies (s' = s \cup \{i\})\). From this definition we can derive that \(w_{s'} > w_s\) (all item weights are positive). So \(g[w_s]\) is found by the UKP1 second loop (lines~\ref{ukp1:main_ext_loop_begin}~to~\ref{ukp1:main_ext_loop_end}) before \(g[w_s']\). We guarantee that the position \(g[w_s]\) will not be skipped by the if statement at line~\ref{ukp1:if_equal_to_zero} because, for any nonempty solution~\(s\), \(p_s > 0\) and \(g[w_s] = p_s\). Then the UKP1 inner loop (lines~\ref{ukp1:main_inner_loop_begin}~to~\ref{ukp1:main_inner_loop_end}) will combine \(s\) with every item making every possible \(s'\) out of it. If there was already a solution with weight \(w_{s'}\), and \(s'\) profit-dominates that solution, then \(g[w_s']\) is updated.

This way, when the outer loop iteration with \(y = w_{s'}-w_{min}\) ends, UKP1 will have computed all the non profit-dominated solutions with weight lesser than or equal to \(w_{s'}\). The final value of \(y\) is \(c-w_{min}\) so, after the last iteration of the outer loop, we will have stored in \(g\) all the non profit-dominated solutions with weight lesser than or equal to~\(c\) (i.e. all the valid non profit-dominated solutions).

Note that \(s'\) can be an invalid solution (\(w_{s'} > c\)). That is not relevant because our theorem states that all valid solutions are present, it does not says that invalid ones aren't.

\subsection{UKP1 Second Theorem: An optimal solution is present in \(g\)}

\begin{corollary}\label{theo:ukp1:opt_is_present}
%After the execution of UKP1 line~\ref{ukp1:main_ext_loop_end}, if \(s\) is an optimal solution, then \(g[w_{s}] = p_{s}\). In mathematical notation: \(\forall s' \in V.~\forall s \in V.~(p_{s'} \leq p_s) \land (g[w_s] = p_s)\).
After the execution of UKP1 line~\ref{ukp1:main_ext_loop_end}, exists at least one optimal solution~\(s^*\) for that \(g[w_{s^*}] = p_{s^*}\). In mathematical notation: \(\forall s \in V.~\exists s^* \in V.~(p_{s} \leq p_{s^*}) \land (g[w_{s^*}] = p_{s^*})\).
\end{corollary}

This corollary is implied by \autoref{theo:ukp1:all_relevant_solutions_present}, that we just proved true. If we look at the mathematical description of both theorems this becomes apparent. The only difference from this corollary and \autoref{theo:ukp1:all_relevant_solutions_present} (\(\forall s \in V.~\exists s^* \in V.~(w_s = w_{s^*}) \land (p_s \leq p_{s^*}) \land (g[w_{s^*}] = p_{s^*})\)) is that this corollary don't state that \((w_s = w_{s*})\). In other words, the theorem states that for every valid solution there's a solution with the same weight and a no lower profit stored in \(g\). The corollary states that for every valid solution there's a solution with a no lower profit stored in \(g\). As, by definition, there's an optimal solution in the set of the valid solutions, and we state that there's a solution with no lower profit stored in \(g\), then we are already stating that there's an optimal solution stored in \(g\).

%We could also have proven this using the first theorem (CITE DOMINANCE THEOREM). It is valid for general dominance, therefore is valid for profit-dominance, that is a special case of general dominance. For every valid non profit-dominated solution~\(s\), \(g[w_s] = p_s\) is true (CITE ALL VALID THEOREM). If an optimal solution was profit-dominated, its dominator is in \(g\) and is also optimal. If an optimal solution wasn't profit-dominated, it's already in \(g\). Anyway, \(g\) contains an optimal solution.

\subsection{UKP1 Third Theorem: \(opt\) value is equal to the optimal solution value}

\begin{theorem}\label{theo:ukp1:opt_variable}
After the execution of UKP1 line~\ref{ukp1:get_opt_loop_end}, the \(opt\) variable value is the optimal solution value. In mathematical notation: \(\exists s \in V.~\forall s' \in V.~(p_s \geq p_{s'}) \land (opt = p_s)\).
\end{theorem}

\vspace{0.3cm}
\noindent
To prove this theorem we will first prove another closely related theorem.
\vspace{0.2cm}

\begin{theorem}\label{theo:opt_sol_weight_range}
If \(s^*\) is an optimal solution, then \(c - w_{min} < w_{s^*} \leq c\). In mathematical notation: \(\forall s^* \in V.~(\nexists s \in V.~p_s > p_{s^*}) \implies ((c - w_{min} < w_{s^*}) \land (w_{s^*} \leq c))\).
\end{theorem}

\vspace{0.3cm}
\noindent
Half of the proof is done by unraveling the optimal solution definition, the other half is done using a proof by contradiction.
\vspace{0.2cm}

\begin{proof}
Any optimal solution \(s^*\) is, by definition, a valid solution, and any valid solution~\(s\) have, by definition, \(w_s \leq c\); therefore any optimal solution \(s^*\) have \(w_{s^*} \leq c\). Now, lets suppose that exists an optimal solution \(s^*\) with \({w_{s^*} \leq c - w_{min}}\); by consequence there exists a valid solution \(s = s^* \cup \{i\}\) where \(w_i = w_{min}\); as all item profits are positive, \(p_s > p_{s^*}\) is guaranteed; if \(s^*\) is an optimal solution then there's no valid solution~\(s\) with \(p_s > p_{s^*}\); it was shown that a solution~\(s\) with this property exists, therefore \(s^*\) can't be an optimal solution; the only thing we assumed was that \(w_{s^*} \leq c - w_{min}\) so we can conclude that a solution can't be optimal and, at the same time, have its weight within this range. As we proved that there are no optimal solutions \(s^*\) with \(w_{s^*} > c\) or \(w_{s^*} \leq (c - w_{min})\), the weight of any optimal solution must be between \((c + w_{min}) + 1\) and \(c\).
\end{proof}

In the light of \autoref{theo:opt_sol_weight_range} we can see that proving \autoref{theo:ukp1:opt_variable} is rather trivial. We know that at least one optimal solution is stored in \(g\) (\autoref{theo:ukp1:opt_is_present}), that the weight of an optimal solution is in the range \(c - w_{min} < w_{s^*} \leq c\) (\autoref{theo:opt_sol_weight_range}), and that an optimal solution has the biggest profit between all valid solutions (by definition), therefore the UKP1 last loop (lines~\ref{ukp1:get_opt_loop_begin}~to~\ref{ukp1:get_opt_loop_end}) must end with the optimal solution value stored in variable \(opt\).

\section{UKP2 -- Pruning symmetry -- Only one way to reach each solution}

\begin{algorithm}[H]
\caption{UKP Two}\label{alg:ukp2}
\begin{algorithmic}[1]
\Procedure{UKP2}{$n, c, w, p, w_{min}, w_{max}$}
  \State \(g \gets\) array of \(c + 1 + (w_{max} - w_{min})\) positions each one initialized with \(0\)\label{ukp2:create_g}
  \State \underline{\(d \gets\) array of \(c + 1 + (w_{max} - w_{min})\) positions each one initialized with \(n\)\label{ukp2:create_d}}
  \State % Blank line
  \For{\(i \gets 1,~n\)}\label{ukp2:begin_trivial_bounds}
    \If{\(g[w_i] < p_i\)}
      \State \(g[w_i] \gets p_i\)
      \State \underline{\(d[w_i] \gets i\)}
    \EndIf
  \EndFor\label{ukp2:end_trivial_bounds}
  \State % Blank line
  \For{\(y \gets w_{min},~c-w_{min}\)}\label{ukp2:main_ext_loop_begin}
    \If{\(g[y] == 0\)}\label{ukp2:if_equal_to_zero}
    	\State \textbf{continue}
    \EndIf
    \State % Blank line
    \For{\(i \gets 1,~\underline{d[y]}\)}\label{ukp2:main_inner_loop_begin}
      \If{\(g[y + w_i] < g[y] + p_i\)}\label{ukp2:if_better_solution_begin}
        \State \(g[y + w_i] \gets g[y] + p_i\)
        \State \(\underline{d[y + w_i] \gets i}\)
      \EndIf\label{ukp2:if_better_solution_end}
    \EndFor\label{ukp2:main_inner_loop_end}
  \EndFor\label{ukp2:main_ext_loop_end}
  \State % Blank line
  \State \(opt \gets 0\)
  \For{\(y \gets c-w_{min}+1,~c\)}\label{ukp2:get_opt_loop_begin}
    \If{\(g[y] > opt\)}\label{ukp2:opt_loop_if}
      \State \(opt \gets g[y]\)
    \EndIf
  \EndFor\label{ukp2:get_opt_loop_end}
\EndProcedure
\end{algorithmic}
\end{algorithm}

The code above is almost equal to the first one presented. The only lines added or changed are the ones underlined. The intent behind those lines is to avoid reaching the same solution by many paths, as this wastes computation and is not necessary for reaching an optimal solution. In the linear programming jargon what those lines do is to eliminate solution symmetry. To prove that the code stands correct we will update the proof of \autoref{theo:ukp1:all_relevant_solutions_present}. But first we will give some explanation for the reasoning behind those changes.

\begin{figure}[H]
\centering
\edef \scale {0.5}
\begin{tikzpicture}[scale=\scale]

\edef \rx {1}
\edef \ry {1}

\edef \origin {(0,0)}
\drawvvector{\origin}{7}{\rx}{\ry}{\scale}
\draw[dashed, ultra thick] ++\origin +(0, 0.5) to [out=135,in=225] +(0, 3.5);
\draw[dashed, ultra thick] ++\origin +(0, 0.5) to [out=135,in=225] +(0, 4.5);

\draw[dashed, thick] ++\origin +(1, 3.5) to [out=45,in=315] +(1, 7.5);
\draw[dashed, thick] ++\origin +(1, 4.5) to [out=45,in=315] +(1, 7.5);

\edef \origin {(3,0)}
\drawvvector{\origin}{7}{\rx}{\ry}{\scale}
\draw[dashed, ultra thick] ++\origin +(0, 0.5) to [out=135,in=225] +(0, 3.5);
\draw[dashed, ultra thick] ++\origin +(0, 0.5) to [out=135,in=225] +(0, 4.5);

\draw[dashed, thick] ++\origin +(1, 4.5) to [out=45,in=315] +(1, 7.5);

\edef \origin {(6,5)}
\drawhvector{\origin}{15}{\rx}{\ry}{\scale}

\draw[dashed, ultra thick] ++\origin +(0.5, 1) to [out=45,in=135] +(4.5, 1);
\draw[dashed, ultra thick] ++\origin +(0.5, 1) to [out=45,in=135] +(5.5, 1);
\draw[dashed, ultra thick] ++\origin +(0.5, 1) to [out=45,in=135] +(6.5, 1);

\draw[dashed, thick] ++\origin +(4.5, 0) to [out=315,in=225] +(9.5, 0);
\draw[dashed, thick] ++\origin +(4.5, 0) to [out=315,in=225] +(10.5, 0);

\draw[dashed, thick] ++\origin +(5.5, 0) to [out=315,in=225] +(9.5, 0);
\draw[dashed, thick] ++\origin +(5.5, 0) to [out=315,in=225] +(11.5, 0);

\draw[dashed, thick] ++\origin +(6.5, 0) to [out=315,in=225] +(10.5, 0);
\draw[dashed, thick] ++\origin +(6.5, 0) to [out=315,in=225] +(11.5, 0);

\draw[dashed] ++\origin +(9.5, 1) to [out=45,in=135] +(15.5, 1);
\draw[dashed] ++\origin +(10.5, 1) to [out=45,in=135] +(15.5, 1);
\draw[dashed] ++\origin +(11.5, 1) to [out=45,in=135] +(15.5, 1);

\def \origin {(6,1.5)}
\drawhvector{\origin}{15}{\rx}{\ry}{\scale}

\draw[dashed, ultra thick] ++\origin +(0.5, 1) to [out=45,in=135] +(4.5, 1);
\draw[dashed, ultra thick] ++\origin +(0.5, 1) to [out=45,in=135] +(5.5, 1);
\draw[dashed, ultra thick] ++\origin +(0.5, 1) to [out=45,in=135] +(6.5, 1);

\draw[dashed, thick] ++\origin +(5.5, 0) to [out=315,in=225] +(9.5, 0);

\draw[dashed, thick] ++\origin +(6.5, 0) to [out=315,in=225] +(10.5, 0);
\draw[dashed, thick] ++\origin +(6.5, 0) to [out=315,in=225] +(11.5, 0);

\draw[dashed] ++\origin +(11.5, 1) to [out=45,in=135] +(15.5, 1);

\end{tikzpicture}
\caption{The solution \(\{2,1\}\) and \(\{2,1\}\) (with \(w_1 = 3\) and \(w_2 = 4\)) are the same. But it can be constructed by two different ways. The solution \(\{3,2,1\}\) can be also constructed by many ways (with \(w_1 = 4\), \(w_2 = 5\) and \(w_3 = 6\)). }
\label{fig:symmetry}
\end{figure}

The UKP1 does not store symmetric solutions. Of many solutions with the same weight only one is stored. As the same solution (reached from different paths) has the same weight, it will be stored only one time and used for creating new solutions only \(n\) times. This avoids creating an exponential number of equivalent solutions (all the permutations of an multiset). The problem with UKP1 is that the solution \(s\) will be stored in \(g[w_s]\) only one time, but it will be computed and compared (as in the line \(g[y + w_i] < g[y] + p_i\)) many times, when only one time was sufficient (for each distinct solution). 

The path for reaching a solution can be defined as a list of the items that compose the solution. To say that UKP computed the \([3,2,1]\) path is to say that: at \(y = w_3\) it computed \(g[y + w_2] < g[y] + p_2\), and at \(y = w_3 + w_2\) it computed \(g[y + w_1] < g[y] + p_1\). The UKP1 compute all the possible paths for a solution \(s\). The UKP2 compute only one path for each solution \(s\). This is done by defining a canonical ordering. Any ordering will suffice, the UKP2 specifically define that we will only add an item to an existing solution if the item index is lower than or equal to the index of the last item added. This is the same as specifying that only the path \([3,2,1]\) will be computed (and that none of the following paths will be computed: \([3,1,2]\), \([2,3,1]\), \([2,1,3]\), \([1,2,3]\) and \([1,3,2]\)). To implement this control we use the \(d\) array. The cell \(d[w_s]\) stores the index of the last item used to make solution \(s\) (\(s\) is the undominated solution of weight \(w_s\)). 

The \(\autoref{fig:symmetry}\) shows a two item and a three item example of the difference between the paths computed on UKP1 and UKP2. We will discuss only the three item example here. Paths as \(\{4,4,4\}\) or \(\{5,5,4\}\) are also generated by UKP1 and UKP2, but aren't show in the figure to not pollute the example. The only paths shown in the figure are the ones that can lead to \(\{3,2,1\}\) or some equivalent solution (i.e. paths that take only one of each item). The top example show the behavior of UKP1, and the bottom one the behavior of UKP2. While UKP1 generates and test all paths that can lead to \(\{3,2,1\}\), UKP2 generate only \(\{3\}\), \(\{3,1\}\), \(\{3,2\}\) and \(\{3,2,1\}\).

\subsection{Updated theorem and proof}

Our following theorem is the same as \autoref{theo:ukp1:all_relevant_solutions_present} except it refers to UKP2.
\vspace{0.3cm}
\begin{theorem}\label{theo:ukp2:all_relevant_solutions_present}
After the execution of UKP2 line~\ref{ukp2:main_ext_loop_end}, for every valid non profit-dominated solution~\(s^*\), the profit value of \(s^*\) is stored in \(g[w_{s^*}]\). In mathematical notation: \(\forall s \in V.~\exists s^* \in V.~(w_s = w_{s^*}) \land (p_s \leq p_{s^*}) \land (g[w_{s^*}] = p_{s^*})\).
\end{theorem}

\noindent
We prove this theorem by structural induction, as we did for \autoref{theo:ukp1:all_relevant_solutions_present} previously.

\subsubsection{Base case: Empty solution and one single item solutions}

The reasoning is the same presented in \autoref{ukp1:empty_solution_proof} and \autoref{ukp1:single_item_solution_proof}. The only difference is that the \(d[w_i]\) is updated with \(i\) (index of the last added item). The empty solution has \(d[0] = n\), but that is irrelevant because we know that no item is used to make the empty solution.

\subsubsection{Induction Hypothesis: \(s\) solutions}

Assume that for some valid nonempty non profit-dominated solution~\(s\), the profit value of \(s\) is stored in \(g[w_s]\), and the lowest index of an item used to make up \(s\) is stored in \(d[w_s]\). Both statements are true for each one of our nonempty base cases.

\subsubsection{Step case: \(s \cup \{i\}\) solutions}

Lets define the function \(min\_ix(s)\) as a function that maps a solution \(s\) to the index of the item with lowest index that compose \(s\) (i.e. \(min\_ix(s)\) return \(i\) where \(i\) is defined as follows \(\forall i' \in s.~\exists i \in s.~i \leq i'\)).

Lets define the \(s'\) solution as \(s' = s \cup \{i\}\) where \(s\) is a valid nonempty non profit-dominated solution with \(w_s \leq c-w_{min}\) and \(i \in I\). Our code only generates \(s'\) if \(i \leq d[w_s]\), but, for any \(s'\), there's a solution \(s\) and an item \(i\) that satisfies that premise. If \(i = min\_ix(s')\), \(s = s' - \{i\}\) and \(y = w_s\), the lines \ref{ukp2:main_ext_loop_begin}~to~\ref{ukp2:main_ext_loop_end} will compute \(s'\). As \(w_s < w_{s'}\) we guarantee that the loop will generate \(s'\) before reaching \(g[w_{s'}]\) (and \(d[w_{s'}]\)).

\section{UKP3 -- Pruning dominated solutions}

\begin{algorithm}
\caption{UKP Three}\label{alg:ukp3}
\begin{algorithmic}[1]
\Procedure{UKP3}{$n, c, w, p, w_{min}, w_{max}$}
  \State \(g \gets\) array of \(c + 1 + (w_{max} - w_{min})\) positions each one initialized with \(0\)\label{ukp3:create_g}
  \State \(d \gets\) array of \(c + 1 + (w_{max} - w_{min})\) positions each one initialized with \(n\)\label{ukp3:create_d}
  \State % Blank line
  \For{\(i \gets 1,~n\)}\label{ukp3:begin_trivial_bounds}
    \If{\(g[w_i] < p_i\)}
      \State \(g[w_i] \gets p_i\)
      \State \(d[w_i] \gets i\)
    \EndIf
  \EndFor\label{ukp3:end_trivial_bounds}
  \State % Blank line
  \State \underline{\(opt \gets 0\)}
  \For{\(y \gets w_{min},~c-w_{min}\)}\label{ukp3:main_ext_loop_begin}
    \If{\underline{\(g[y] \leq opt\)}}\label{ukp3:if_no_greater_than_opt}
    	\State \textbf{continue}
    \EndIf
    \State % Blank line
    \State \underline{\(opt \gets g[y]\)}
    \State % Blank line
    \For{\(i \gets 1,~d[y]\)}\label{ukp3:main_inner_loop_begin}
      \If{\(g[y + w_i] < g[y] + p_i\)}\label{ukp3:if_better_solution_begin}
        \State \(g[y + w_i] \gets g[y] + p_i\)
        \State \(d[y + w_i] \gets i\)
      \EndIf\label{ukp3:if_better_solution_end}
    \EndFor\label{ukp3:main_inner_loop_end}
  \EndFor\label{ukp3:main_ext_loop_end}
  \State % Blank line
  \State \sout{\(opt \gets 0\)}
  \For{\(y \gets c-w_{min}+1,~c\)}\label{ukp3:get_opt_loop_begin}
    \If{\(g[y] > opt\)}\label{ukp3:opt_loop_if}
      \State \(opt \gets g[y]\)
    \EndIf
  \EndFor\label{ukp3:get_opt_loop_end}
\EndProcedure
\end{algorithmic}
\end{algorithm}

The UKP1 and UKP2 compute all valid solutions and store only the non profit-dominated ones. The UKP3 doesn't compute all valid solutions. If the UKP3 finds that a solution \(s\) that is dominated by a solution \(s'\), it wont create any new solutions from \(s\). UKP2 guarantee that exists only one path for each distinct solution, UKP3 keep that guarantee. In UKP3, any solution that is a superset of a dominated solution wont be computed anymore (the path will be interrupted at that point). Solutions that are an superset of \(s\) and are optimal can exist but, if this happens, there will be an optimal solution that is a superset of \(s'\) too\footnote{This was already proved on \autoref{theo:dominance}.}. If UKP3 skips a dominated solution \(s\) there's the guarantee that a solution that dominates \(s\) wasn't skipped.

Lets examine the first iteration of UKP3 outer loop (lines~\ref{ukp3:main_ext_loop_begin}~to~\ref{ukp3:main_ext_loop_end}). The \(opt\) variable is zero, but any item profit is positive, therefore \(g[w_{min}] > opt\). In the first iteration the if statement in line \ref{ukp3:if_no_greater_than_opt} will eval to false. The \(opt\) value will be updated to \(g[w_{min}]\). In all iterations afterwards, \(opt > 0\) is true, and because of that UKP3 will skip the \(g[y] = 0\) positions as UKP1 and UKP2 already did. More than that, it will skip any dominated solution. If \(g[y] \leq opt\) and \(g[y] > 0\), there exists a previous solution \(s'\), where \(w_{s'} < y\) and \(p_{s'} = opt\), and exists a current solution \(s\), where \(w_s = y\) and \(p_s \leq opt\), therefore, by definition, \(s\) is dominated by \(s'\) (i.e. \((w_{s'} < y = w_s) \land (p_{s'} = opt \geq p_s)\)).

We think that is clear that the UKP3 only doesn't compute solutions that are a superset of a dominated solution and, as already proven by \autoref{theo:dominance}, discarding only solutions with this property can't eliminate all optimal solutions. Therefore the changes introduced by UKP3 don't affect the correctness of the algorithm, and UKP3 stands correct.

\section{UKP4 -- Knowing when to stop}

\begin{algorithm}
\caption{UKP Four}\label{alg:ukp4}
\begin{algorithmic}[1]
\Procedure{UKP4}{$n, c, w, p, w_{min}, w_{max}$}
  \State \(g \gets\) array of \(c + 1 + (w_{max} - w_{min})\) positions each one initialized with \(0\)\label{ukp4:create_g}
  \State \(d \gets\) array of \(c + 1 + (w_{max} - w_{min})\) positions each one initialized with \(n\)\label{ukp4:create_d}
  \State \(y' \gets 0\)
  \State % Blank line
  \For{\(i \gets 1,~n\)}\label{ukp4:begin_trivial_bounds}
    \If{\(g[w_i] < p_i\)}
      \State \(g[w_i] \gets p_i\)
      \State \(d[w_i] \gets i\)
      \If{\underline{\(i \neq 1 \land w_i > y'\)}}
      	\State \underline{\(y' \gets w_i\)}
      \EndIf
    \EndIf
  \EndFor\label{ukp4:end_trivial_bounds}
  \State % Blank line
  \State \(opt \gets 0\)
  \For{\(y \gets w_{min},~c-w_{min}\)}\label{ukp4:main_ext_loop_begin}
    \If{\(g[y] \leq opt\)}\label{ukp4:if_no_greater_than_opt}
    	\State \textbf{continue}
    \EndIf
    \If{\(y > y'\)}\label{ukp4:if_greater_than_yprime}
    	\State \textbf{break}
    \EndIf
    \State % Blank line
    \State \(opt \gets g[y]\)
    \State % Blank line
    \For{\(i \gets 1,~d[y]\)}\label{ukp4:main_inner_loop_begin}
      \If{\(g[y + w_i] < g[y] + p_i\)}\label{ukp4:if_better_solution_begin}
        \State \(g[y + w_i] \gets g[y] + p_i\)
        \State \(d[y + w_i] \gets i\)
        \If{\underline{\(i \neq 1 \land y + w_i > y'\)}}
	  \State \underline{\(y' \gets y + w_i\)}
        \EndIf
      \EndIf\label{ukp4:if_better_solution_end}
    \EndFor\label{ukp4:main_inner_loop_end}
  \EndFor\label{ukp4:main_ext_loop_end}
  \State % Blank line
  \If{\(y' < c - w_{min}\)}\label{ukp4:check_yprime}
  	
  \Else
    \For{\(y \gets c-w_{min}+1,~c\)}\label{ukp4:get_opt_loop_begin}
      \If{\(g[y] > opt\)}\label{ukp4:opt_loop_if}
        \State \(opt \gets g[y]\)
      \EndIf
    \EndFor\label{ukp4:get_opt_loop_end}
  \EndIf
\EndProcedure
\end{algorithmic}
\end{algorithm}


%The algorithm with the periodic check change:
%	* We prove that if a condition defined by us is met, we can stop the computation and calculate the optimal solution value from a simple formula in O(1) time.

%\section{UKP5 -- Recovering an optimal solution}

\end{document} 

\bibliographystyle{sbc.bst}
\bibliography{sbc-template.bib}

\end{document}

