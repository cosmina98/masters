\documentclass{beamer}

\usepackage[T1]{fontenc}
\usepackage[brazil]{babel}
\usepackage[utf8]{inputenc}

% Choose the Inf theme
\usetheme{Inf}

% Define the title with \title[short title]{long title}
% Short title is optional
\title[UKP5: a New Algorithm for the UKP]
      {UKP5: a New Algorithm for the Unbounded Knapsack Problem}

% Optional subtitle
\subtitle{15th International Symposium on Experimental Algorithms - SEA 2016}

\date{Wednesday, June 8 2016}

% Author information
\author{Henrique Becker}
\institute{Federal University of Rio Grande do Sul (UFRGS),\\
Porto Alegre, Brazil\\
hbecker@inf.ufrgs.br\\
\url{http://ppgc.inf.ufrgs.br/}
}

\begin{document}

% Command to create title page
\InfTitlePage

\begin{frame}
  \frametitle{Outline}
  \tableofcontents
\end{frame}

\section{Detalhes}

\frame{
    \frametitle{História}

    \begin{itemize}
        \item O Instituto de Informática mudou de logotipo quando completou 20 anos
        \item Entretanto os templates para slides continuaram os mesmos
        \item Um manual de identidade visual veio junto com o artigo
        \item Esse template que você vê é baseado nele
    \end{itemize}
}

\frame{
    \frametitle{Listas}

    \begin{enumerate}
        \item Listas não numeradas você já viu no slide anterior
        \item Listas numeradas também são possíveis
        \begin{itemize}
            \item Assim como ambos os tipos em vários níveis
        \end{itemize}
    \end{enumerate}
}

\section{Outra seção}
\subsection{Primeira subseção}

\frame{
    \frametitle{Seções}

    \begin{itemize}
        \item O número a esquerda do cabeçalho marca a seção atual
        \item Logo a direita ele é seguido por:
        \begin{itemize}
            \item nome da seção
            \item subseção
            \item título do slide
        \end{itemize}
    \end{itemize}
}

\section{Blocos}

\begin{frame}[plain]
 \sectionpage
\end{frame}

\frame{
    \frametitle{Teoremas}

    \begin{theorem}
        foo
    \end{theorem}
    \begin{proof}
        bar
    \end{proof}
}

\frame{
    \frametitle{Exemplos}

    \begin{exampleblock}{exemplo}
        foo
    \end{exampleblock}
}

\frame{
    \frametitle{Alertas}

    \begin{alertblock}{alerta}
        foo
    \end{alertblock}
}

\section*{}

\begin{frame}
    \frametitle{Obrigado!}
    \InfContacts
\end{frame}

\end{document}
