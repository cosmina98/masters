An empirical analysis of exact algorithms for the unbounded knapsack problem was carried out.
The experiments included seven algorithms from the literature and more than ten thousand instances.
The terminating step-off, a dynamic programming algorithm from 1966, was found to have the lowest mean time to solve the most recent benchmark from the literature.
The threshold and collective dominance are properties of the unbounded knapsack problem first discussed in 1998, and the terminating step-off did not implement them, but has an alternative mechanism for dealing with dominance that seem competitive with the application of such dominances.
The authors present a new class of instances.
This class of instances favors the branch-and-bound approach over the dynamic programming approach but do not have high amounts of simple, multiple and collective dominated items.
The pricing subproblems from solving hard cutting stock problems with column generation can cause branch-and-bound algorithms to display worst-case times. %seem to be best solved with dynamic programming as the branch-and-bound algorithms tested displayed worst-case times.
The definition of which instances are of interest defines which algorithm is considered the best.
The code used for solving the unbounded knapsack problem and for instance generation are available online.

