This work presents an empirical analysis of exact algorithms for the unbounded knapsack problem, which includes seven algorithms from the literature, two commercial solvers, and more than ten thousand instances.
The terminating step-off, a dynamic programming algorithm from 1966, presented the lowest mean time to solve the most recent benchmark from the literature.
The threshold and collective dominances are properties of the unbounded knapsack problem first discussed in 1998, and exploited by the current state-of-the-art algorithms. 
The terminating step-off algorithm did not exploit such dominances, but has an alternative mechanism for dealing with dominances which does not explicitly exploits collective and threshold dominances.
Also, the pricing subproblems found when solving hard cutting stock problems with column generation can cause branch-and-bound algorithms to display worst-case times.
The authors present a new class of instances which favors the branch-and-bound approach over the dynamic programming approach but do not have high amounts of simple, multiple and collective dominated items.
This behaviour illustrates how the definition of hard instances changes among algorithm approachs.
The codes used for solving the unbounded knapsack problem and for instance generation are all available online.
